\newpage
\section{Билет 4. Тензор скоростей деформации. Его связь с полем скоростей. Кинематический смысл компонент тензора скоростей деформации. Дивиргенция скорости.}

\begin{center}
	\textit{\underline{Определение}}
\end{center}

Введем тензор скоростей деформаций. Используем Лагранжеву систему координат $\xi^{i}$, рассмотрим точку и небольшую ее окрестность. Рассмотрим малые деформации, которые произошли с этой окрестностью за время $\Delta t$. Для таких деформаций начальным является момент времени $t$, а конечным $t + \Delta t$. Тогда тензор деформаций примет вид: 

$$
\Delta \varepsilon_{ij} = \varepsilon_{ij} (t + \Delta t, \xi^i) - \varepsilon_{ij} (t, \xi^i) = \frac{1}{2} \left(\hat g_{ij}(t + \Delta t, \xi) - g_{ij}^0(\xi)  \right) - \frac{1}{2}
\left(\hat g_{ij}(t, \xi) - g_{ij}^0(\xi) \right) = \frac{1}{2} \left(\hat g_{ij}(t + \Delta t, \xi) - \hat g_{ij}(t, \xi)  \right)
$$

Определение: Рассмотрим предел отношения приращения деформаций к промежутку времени: $lim_{\Delta t \rightarrow 0} \frac{\Delta \varepsilon}{\Delta t} = e_{ij}$,
где $e_{ij}$ - компоненты тензора скоростей деформаций.

Рассмотрим ранее расписанное приращение $\Delta \varepsilon_{ij}$ и подставим в его определение тензора скоростей деформаций. В итоге получим:

$$e_{ij} = \frac{d \hat g_{ij}}{d t}$$

\begin{center}
	\textit{\underline{Выражение компонент тензора через компоненты вектора скорости}}
\end{center}
Вспомним, что $\varepsilon_{ij} = \frac{1}{2} \left(\nabla_{i}w_{j} + \nabla_{j}w_{i} \right)$ (Или в нотации частных производных: $\frac{1}{2} \left(\partial_{i}w_{j} + \partial_{j}w_{i} \right)$). Пусть вектор перемещений $\overrightarrow{w} = \overrightarrow{v} \Delta t$, тогда имеем:

$$\Delta \varepsilon_{ij} = \frac{1}{2} \left(\nabla_{i}\Delta w_{j} + \nabla_{j}\Delta w_{i} \right)$$

Так как $v_{i} = \lim\limits_{\Delta t \rightarrow 0} \frac{\Delta w_{i}}{\Delta t}$, то получаем:

$$e_{ij} = \frac{1}{2} \left(\nabla_{i}v_{j} + \nabla_{j}v_{i}\right)$$

\begin{center}
	\textit{\underline{Кинематический смысл}}
\end{center}
Механический смысл компонент тензора скоростей деформаций в декартовой системе следует из формулы $e_{ij} = \lim\limits_{\Delta t \rightarrow 0} \frac{\varepsilon_{ij}}{\Delta t}$. Ранее было показано, что диагональные компоненты тензора малых деформаций равны коэффициентам относительного удлинения отрезков, лежавших до деформации вдоль соответствующих осей (если система координат декартова). Следовательно, механический смысл диагональных компонент тензора скоростей деформаций - скорость удлинения (сокращения) отрезка. Аналогично, вне диагональные элементы - половина скоростей изменения углов между отрезками. 

\begin{center}
	\textit{\underline{Дивергенция скорости}}
\end{center}
Так как $\varepsilon_{ii} = \frac{1}{2} \left(\nabla_{i}v_{i} + \nabla_{i}v_{i} \right) = \nabla_{i}v_{i}$
$$div \overrightarrow{v} = \sum\limits_{i} \nabla_{i} v^i = \sum\limits_{i} \varepsilon^{i}_{i}$$