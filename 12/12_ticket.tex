\newpage
\section{Билет 12. Уравнения движения идеальной жидкости в форме Громеки-Лемба. Интегралы Бернулли и Коши- Лагранжа.}

\begin{center}
	\textit{\underline{Уравнение Эйлера в форме Громеки-Лэмба}}
\end{center}
Рассмотрим систему уравнений Эйлера.
\begin{center}
	$\begin{cases}
		\frac{\partial u}{\partial t} + u\frac{\partial u}{\partial x} + v\frac{\partial u}{\partial y} + w\frac{\partial u}{\partial z} = F_x - \frac{1}{\rho}\frac{\partial p}{\partial x}\\
		\frac{\partial v}{\partial t} + u\frac{\partial v}{\partial x} + v\frac{\partial v}{\partial y} + w\frac{\partial v}{\partial z} = F_Y - \frac{1}{\rho}\frac{\partial p}{\partial y}\\
		\frac{\partial w}{\partial t} + u\frac{\partial w}{\partial x} + v\frac{\partial w}{\partial y} + w\frac{\partial w}{\partial z} = F_z - \frac{1}{\rho}\frac{\partial p}{\partial z}
	\end{cases}$
\end{center}
Запишем эти уравнения в несколько другом виде. Легко видеть, что усорение всегда возможно написать в следующей форме:
\begin{center}
	$\frac{d \vec{v}}{dt} = \frac{\partial \vec{v}}{\partial t} + grad(\frac{\vec{v}^2}{2}) + 2 \omega \times \vec{v}$
\end{center}
Где $\omega$ - вектор вихря.

Пояснение: используя декартову систему координат, для проекции ускорения на ось x имеем:
\begin{center}
	$\frac{du}{dt} = \frac{\partial u}{\partial t} + u \frac{\partial u}{\partial x} + v \frac{\partial u}{\partial y} + w \frac{\partial u}{\partial z} = \frac{\partial u}{\partial t} + \frac{1}{2}\frac{\partial}{\partial x}\left(u^2 + v^2 + w^2\right) - (\frac{\partial v}{\partial x} - \frac{\partial u}{\partial y}) v + (\frac{\partial u}{\partial z} - \frac{\partial w}{\partial x})w = \frac{\partial u}{\partial x} + \frac{1}{2}\frac{\partial \vec{v}^2}{\partial x} + 2(\omega_yw - \omega_zv) = \frac{\partial u}{\partial t} + \frac{1}{2}\frac{\partial \vec{v}^2}{\partial x} + 2(\omega \times v)_x$
\end{center}

Производя аналогичные вычисления для других осей и подставляя в уравнения Эйлера, получаем следующие уравнения.
\begin{center}
	$\frac{\partial \vec{v}}{\partial t} + \frac{1}{2}grad(\vec{v}^2) + 2(\omega\times \vec{v}) = F - \frac{1}{\rho}grad(p)$
\end{center}
Получаем \underline{Уравнения движения в форме Громека-Лэмба}
\begin{center}
	\textit{\underline{Интеграл Бернулли}}
\end{center}
Запишем уравнения движения эйлера в форме Громеки-Лэмба.
\begin{center}
	$\frac{\partial \vec{v}}{\partial t} + \frac{1}{2}grad(\vec{v}^2) + 2(\omega\times \vec{v}) = F - \frac{1}{\rho}grad(p)$
\end{center}
Будем считать что мы рассматриваем движение идеального газа или жидкости в случае установившихся движений. Поскольку движение установшееся, то имеем что $\frac{\partial \vec{v}}{\partial t} = 0$. Помимо этого предположим что внешние силы обладают потенциалом $F = grad(U)$.
Введем в потоке жидкости кривую L и введем вдоль нее направление отчета длины l, начиная от некоторой точки O. Заданием l будут фиксироваться точки на L. Через dl обозначим элемент касательной к линии L в произвольной точке M. С учетом сделанных предположений получаем:
\begin{center}
	$\frac{\partial}{\partial l}\left(\frac{v^2}{2}\right) + \frac{1}{\rho} \frac{\partial p}{\partial l} - \frac{\partial U}{l} = -2(w\times v)$
\end{center}
Вдоль этой линии плотность и давление являются функциями от длины l. Причем они будут разные для различных линий L. Поэтому имеем.
\begin{center}
	$p = p(l,L) \ \rho = \rho(l,L)$
\end{center}
Очевидно, что вдоль данной линии плотность можно считать функцией давления:
\begin{center}
	$\rho = \rho(p,L)$
\end{center}
И можно всегда ввести функцию давления P
\begin{center}
	$P = P(p,L) = \int_{p_1}^p\frac{dp}{\rho(p,L)}, p_1 = const$
\end{center}
так, что 
\begin{center}
	$\frac{1}{\rho}\frac{\partial p}{\partial l} = \frac{\partial P}{\partial l}$
\end{center}
Используя уравнение состояния, можно получить
\begin{center}
	$P(p,L) = \frac{\gamma}{\gamma - 1}\frac{p}{\rho} + const$
\end{center}

Тогда наше уравнение преобразуется к типу
\begin{center}
	$\frac{\partial}{\partial l}[ \frac{v^2}{2} + P(p,L) - U] = -2(w\times v)_l$
\end{center}
Теперь, пусть L - линия тока. Но тогда правая часть обратиться в нуль, поскольку $(w\times v)$ перпендикулярен линии тока. Аналогично получится если L - линия вихря.


В итоге получаем 
\begin{center}
	$\frac{v^2}{2} + P(p,L) - U = i^*(L)$
\end{center}
Если P - известная функция, то последнее выражение является первым интегралом уравнения движения идеальной жидкости, и называется \underline{интегралом Бернулли}
\begin{center}
	\textit{\underline{Интеграл Коши-Лагранжа}}
\end{center}
Рассмотрим движение идеальной жидкости. Рассмотрим уравнения движения в форме Громеки-Лэмба.
\begin{center}
	$\frac{\partial v}{\partial t} + \frac{1}{2}grad(v^2) + 2(\omega\times v) = F - \frac{1}{\rho}grad(p)$
\end{center}
Пусть движение потенциально ($\omega = 0,\, \vec{v} = grad(\phi)$), имеет место баротропия $p = p(\rho)$ а значит, можно ввести единую для всего потока функцию давления:
\begin{center}
	$P(p) = \int \frac{dp}{\rho(p)}, \frac{1}{\rho}grad(p) = grad(P)$
\end{center}
При этих предположениях, уравнение Громеки-Лэмба записывается в виде
\begin{center}
	$grad(\frac{\partial\phi}{\partial t} + \frac{\vec{v}^2}{2} + P) = F$
\end{center}
В итоге получаем:
\begin{center}
	$\frac{\partial\phi}{\partial t} + \frac{\vec{v}^2}{2} + P - U = f(t)$
\end{center}
где $f(t)$ - некоторая произвольная функция от времени.
Последнее сотношение, выполняющееся для всех точек области потенциального движения называется \underline{интегрaлом Коши-Лагранжа}
